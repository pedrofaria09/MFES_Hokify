\begin{vdmpp}[breaklines=true]
class Utilizador
types
-- TODO Define types here
 public String = seq of char;
 public Sexo = <Masculino> | <Feminino>;
 public Escolaridade = <Secundario> | <Licenciatura> | <Mestrado> | <Doutoramento>;
 public Interesses = set of String;
 public Skills = set of String;
 
values
-- TODO Define values here

instance variables
-- TODO Define instance variables here
 private nome: String;
 private email: String;
 private sexo: Sexo;
 private escolaridade: Escolaridade;
 private idade: nat1;
 private telefone: nat1;
 private pais: String;
 private localidade: String;
 private interesses: Interesses := {};
(*@
\label{Utilizador:24}
@*)
 private skills: Skills := {};
 
operations
-- TODO Define operations here

 --Construtor
 public Utilizador: String * String * Sexo * Escolaridade * String * String * nat1 * nat1 ==> Utilizador
 Utilizador(nm,emailC,sexoC,escolaridadeC,paisC,localidadeC,idadeC,telefoneC) == (
  nome := nm;
  email := emailC;
  sexo := sexoC;
  escolaridade := escolaridadeC;
  idade := idadeC;
  telefone := telefoneC;
(*@
\label{addInteresse:38}
@*)
  pais := paisC;
  localidade := localidadeC;
  return self
 )
 post interesses = {} and
   skills = {} and
(*@
\label{removeInteresse:44}
@*)
   nome = nm and
   email = emailC and
   sexo = sexoC and
   idade = idadeC and
   telefone = telefoneC and
   pais = paisC and
(*@
\label{getInteresses:50}
@*)
   localidade = localidadeC;
 
 -- Adicionar interesses
 public addInteresse: String ==> ()
 addInteresse(String) == interesses := interesses union {String}
 pre String not in set interesses
 post interesses = interesses~ union {String};
(*@
\label{addSkills:57}
@*)
 
 -- Remover interesses
 public removeInteresse: String ==> ()
 removeInteresse(String) == interesses := interesses \ {String}
 pre String in set interesses
 post interesses = interesses~ \ {String};
(*@
\label{removeSkills:63}
@*)
 
 -- Retorna os interesses
 public pure getInteresses : () ==> Interesses
 getInteresses() ==
 (
  return interesses;
(*@
\label{getSkills:69}
@*)
 );
 
 -- Adicionar skills
 public addSkills: String ==> ()
 addSkills(String) == skills := skills union {String}
 pre String not in set skills
 post skills = skills~ union {String};
(*@
\label{getNome:76}
@*)
 
 -- Remover skills
 public removeSkills: String ==> ()
 removeSkills(String) == skills := skills \ {String}
 pre String in set skills
 post skills = skills~ \ {String};
 
 -- Retorna as skills
 public pure getSkills : () ==> Skills
 getSkills() ==
(*@
\label{getIdade:86}
@*)
 (
  return skills;
 );
 
 -- Retorna o nome
 public pure getNome : () ==> String
 getNome() ==
(*@
\label{getTelefone:93}
@*)
 (
  return nome;
 );
 
 -- Retorna a idade
 public pure getIdade : () ==> nat1
 getIdade() ==
(*@
\label{getSexo:100}
@*)
 (
  return idade;
 );
(*@
\label{getEmail:103}
@*)
 
 -- Retorna o telefone
 public pure getTelefone : () ==> nat1
 getTelefone() ==
 (
  return telefone;
 );
 
 -- Retorna o email
 public pure getEmail : () ==> String
 getEmail() ==
 (
  return email;
 );
(*@
\label{setNome:117}
@*)
 
 -- Retorna o sexo
 public pure getSexo : () ==> Sexo
 getSexo() ==
 (
  return sexo;
(*@
\label{setEmail:123}
@*)
(*@
\label{getPais:123}
@*)
 );
 
 -- Retorna o pais
 public pure getPais : () ==> String
 getPais() ==
 (
(*@
\label{setTelefone:129}
@*)
  return pais;
(*@
\label{getLocalidade:130}
@*)
 );
 
 -- Retorna a localidade
 public pure getLocalidade : () ==> String
 getLocalidade() ==
 (
  return localidade;
 );
 
 -- Retorna a escolaridade
(*@
\label{getEscolaridade:140}
@*)
 public pure getEscolaridade : () ==> Escolaridade
 getEscolaridade() ==
 (
  return escolaridade;
 );
 
 -- Editar Nome
 public setNome: String ==> ()
 setNome(newName) == nome := newName
 pre newName <> undefined
 post nome = newName;
 
 -- Editar Email
 public setEmail: String ==> ()
 setEmail(newEmail) == email := newEmail
 pre newEmail <> undefined
 post email = newEmail;
 
(*@
\label{setPais:158}
@*)
 -- Editar Telefone
 public setTelefone: nat1 ==> ()
 setTelefone(newTelefone) == telefone := newTelefone
 pre newTelefone <> undefined
 post telefone = newTelefone;

(*@
\label{setLocalidade:164}
@*)
 -- Editar Pais
 public setPais: String ==> ()
 setPais(newPais) == pais := newPais
 pre newPais <> undefined
 post pais = newPais;
 
(*@
\label{setEscolaridade:170}
@*)
 -- Editar Localidade
 public setLocalidade: String ==> ()
 setLocalidade(newLocalidade) == localidade := newLocalidade
 pre newLocalidade <> undefined
 post localidade = newLocalidade;
 
 -- Editar Escolaridade
 public setEscolaridade: Escolaridade ==> ()
 setEscolaridade(newEscolaridade) == escolaridade := newEscolaridade
 pre newEscolaridade <> undefined
 post escolaridade = newEscolaridade;
 
functions
-- TODO Define functiones here

traces
-- TODO Define Combinatorial Test Traces here

end Utilizador
\end{vdmpp}
\bigskip
\begin{longtable}{|l|r|r|r|}
\hline
Function or operation & Line & Coverage & Calls \\
\hline
\hline
\hyperref[Utilizador:24]{Utilizador} & 24&100.0\% & 4 \\
\hline
\hyperref[addInteresse:38]{addInteresse} & 38&100.0\% & 4 \\
\hline
\hyperref[addSkills:57]{addSkills} & 57&100.0\% & 4 \\
\hline
\hyperref[getEmail:103]{getEmail} & 103&100.0\% & 4 \\
\hline
\hyperref[getEscolaridade:140]{getEscolaridade} & 140&100.0\% & 2 \\
\hline
\hyperref[getIdade:86]{getIdade} & 86&100.0\% & 2 \\
\hline
\hyperref[getInteresses:50]{getInteresses} & 50&100.0\% & 4 \\
\hline
\hyperref[getLocalidade:130]{getLocalidade} & 130&100.0\% & 3 \\
\hline
\hyperref[getNome:76]{getNome} & 76&100.0\% & 4 \\
\hline
\hyperref[getPais:123]{getPais} & 123&100.0\% & 3 \\
\hline
\hyperref[getSexo:100]{getSexo} & 100&100.0\% & 2 \\
\hline
\hyperref[getSkills:69]{getSkills} & 69&100.0\% & 4 \\
\hline
\hyperref[getTelefone:93]{getTelefone} & 93&100.0\% & 4 \\
\hline
\hyperref[removeInteresse:44]{removeInteresse} & 44&100.0\% & 2 \\
\hline
\hyperref[removeSkills:63]{removeSkills} & 63&100.0\% & 2 \\
\hline
\hyperref[setEmail:123]{setEmail} & 123&100.0\% & 1 \\
\hline
\hyperref[setEscolaridade:170]{setEscolaridade} & 170&100.0\% & 1 \\
\hline
\hyperref[setLocalidade:164]{setLocalidade} & 164&100.0\% & 1 \\
\hline
\hyperref[setNome:117]{setNome} & 117&100.0\% & 1 \\
\hline
\hyperref[setPais:158]{setPais} & 158&100.0\% & 1 \\
\hline
\hyperref[setTelefone:129]{setTelefone} & 129&100.0\% & 1 \\
\hline
\hline
Utilizador.vdmpp & & 100.0\% & 54 \\
\hline
\end{longtable}

