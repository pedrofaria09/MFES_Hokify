\begin{vdmpp}[breaklines=true]
class Utilizador
types
-- TODO Define types here
 public String = seq of char;
 public Sexo = <Masculino> | <Feminino>;
 public Escolaridade = <Secundario> | <Licenciatura> | <Mestrado> | <Doutoramento>;
 public ListaEscolaridade = set of Escolaridade;
 public Interesses = set of String;
 public Skills = set of String;
 public Date :: year : nat month: nat1 day : nat 
  inv mk_Date(y,m,d) == m <= 12 and d <= DaysOfMonth(m,y);
 
values
-- TODO Define values here

instance variables
-- TODO Define instance variables here
 private nome: seq of char;
 private email: seq of char;
 private sexo: Sexo;
 private escolaridade: Escolaridade;
 private idade: nat1;
 private telefone: nat1;
 private pais: seq of char;
 private localidade: seq of char;
 private interesses: set of String := {};
 private skills: set of String := {};
 private listaEscolaridade: ListaEscolaridade := {};
 private dataNascimento : Date;
 
operations
(*@
\label{Utilizador:32}
@*)
-- TODO Define operations here

 --Construtor
 public Utilizador: seq of char * seq of char * Sexo * Escolaridade * seq of char * seq of char * nat1 * nat1 * nat * nat1 * nat1 ==> Utilizador
 Utilizador(nm,emailC,sexoC,escolaridadeC,paisC,localidadeC,idadeC,telefoneC,year,month,day) == (
  nome := nm;
  email := emailC;
  sexo := sexoC;
  escolaridade := escolaridadeC;
  listaEscolaridades(escolaridade);
  idade := idadeC;
  telefone := telefoneC;
  pais := paisC;
  localidade := localidadeC;
  dataNascimento := mk_Date(year, month, day);
  return self
 )
 pre verificarEscolaridade(escolaridadeC, idadeC)
 post interesses = {} and
   skills = {} and
   nome = nm and
   email = emailC and
   sexo = sexoC and
   idade = idadeC and
   telefone = telefoneC and
(*@
\label{verificarEscolaridade:57}
@*)
   pais = paisC and
   localidade = localidadeC;
 
 -- Verifica se o utilizador tem idade para a escolaridade
 public pure verificarEscolaridade: Escolaridade * nat1 ==> bool
 verificarEscolaridade(escola, idd)==(
 if (escola = <Secundario> and idd > 17) then(
  return true;
 )elseif(escola = <Licenciatura> and idd > 20) then(
  return true;
 )elseif(escola = <Mestrado> and idd > 22) then(
  return true;
 )elseif(escola = <Doutoramento> and idd > 24) then(
  return true;
 )else
(*@
\label{listaEscolaridades:72}
@*)
  return false;
 )pre escolaridade <> undefined;
 
 -- Adiciona uma lista com as escolaridades no qual possui
 public listaEscolaridades: Escolaridade ==> ()
 listaEscolaridades(tipo)==(
 if tipo = <Doutoramento> then (
  listaEscolaridade := listaEscolaridade union {<Secundario>};
  listaEscolaridade := listaEscolaridade union {<Licenciatura>};
  listaEscolaridade := listaEscolaridade union {<Mestrado>};
  listaEscolaridade := listaEscolaridade union {<Doutoramento>};
 )elseif tipo = <Mestrado> then(
  listaEscolaridade := listaEscolaridade union {<Secundario>};
  listaEscolaridade := listaEscolaridade union {<Licenciatura>};
  listaEscolaridade := listaEscolaridade union {<Mestrado>};
 )elseif tipo = <Licenciatura> then(
  listaEscolaridade := listaEscolaridade union {<Secundario>};
  listaEscolaridade := listaEscolaridade union {<Licenciatura>};
 )else(
  listaEscolaridade := listaEscolaridade union {<Secundario>};
(*@
\label{addInteresse:92}
@*)
 );
 )pre tipo <> undefined;
 
 -- Adicionar interesses
 public addInteresse: seq of char ==> ()
 addInteresse(String) == interesses := interesses union {String}
(*@
\label{removeInteresse:98}
@*)
 pre String not in set interesses
 post interesses = interesses~ union {String};
 
 -- Remover interesses
 public removeInteresse: seq of char ==> ()
 removeInteresse(String) == interesses := interesses \ {String}
(*@
\label{getInteresses:104}
@*)
 pre String in set interesses
 post interesses = interesses~ \ {String};
 
 -- Retorna os interesses
 public pure getInteresses : () ==> set of String
 getInteresses() ==
 (
(*@
\label{addSkills:111}
@*)
  return interesses;
 );
 
 -- Adicionar skills
 public addSkills: seq of char ==> ()
 addSkills(String) == skills := skills union {String}
(*@
\label{removeSkills:117}
@*)
 pre String not in set skills
 post skills = skills~ union {String};
 
 -- Remover skills
 public removeSkills: seq of char ==> ()
 removeSkills(String) == skills := skills \ {String}
(*@
\label{getSkills:123}
@*)
 pre String in set skills
 post skills = skills~ \ {String};
 
 -- Retorna as skills
 public pure getSkills : () ==> set of String
 getSkills() ==
 (
(*@
\label{getNome:130}
@*)
  return skills;
 );
 
 -- Retorna o nome
 public pure getNome : () ==> seq of char
 getNome() ==
 (
(*@
\label{getIdade:137}
@*)
  return nome;
 );
 
 -- Retorna a idade
 public pure getIdade : () ==> nat1
 getIdade() ==
 (
(*@
\label{getTelefone:144}
@*)
  return idade;
 );
 
 -- Retorna o telefone
 public pure getTelefone : () ==> nat1
 getTelefone() ==
 (
(*@
\label{getEmail:151}
@*)
  return telefone;
 );
 
 -- Retorna o email
 public pure getEmail : () ==> seq of char
 getEmail() ==
 (
(*@
\label{getSexo:158}
@*)
  return email;
 );
 
 -- Retorna o sexo
 public pure getSexo : () ==> Sexo
 getSexo() ==
 (
(*@
\label{getPais:165}
@*)
  return sexo;
 );
 
 -- Retorna o pais
 public pure getPais : () ==> seq of char
 getPais() ==
 (
(*@
\label{getLocalidade:172}
@*)
  return pais;
 );
 
 -- Retorna a localidade
 public pure getLocalidade : () ==> seq of char
 getLocalidade() ==
 (
(*@
\label{getEscolaridade:179}
@*)
  return localidade;
 );
 
 -- Retorna a escolaridade
 public pure getEscolaridade : () ==> Escolaridade
 getEscolaridade() ==
 (
(*@
\label{getlistaEscolaridades:186}
@*)
  return escolaridade;
 );
 
 -- Retorna o listaEscolaridades
 public pure getlistaEscolaridades : () ==> ListaEscolaridade
 getlistaEscolaridades() ==
 (
(*@
\label{setNome:193}
@*)
  return listaEscolaridade;
 );
 
 -- Retorna a data de nascimento
(*@
\label{getDataNasccimento:197}
@*)
 public pure getDataNasccimento : () ==> Date
 getDataNasccimento() ==
(*@
\label{setEmail:199}
@*)
 (
  return dataNascimento;
 );
 
 -- Editar Nome
 public setNome: seq of char ==> ()
(*@
\label{setTelefone:205}
@*)
 setNome(newName) == nome := newName
 pre newName <> undefined
 post nome = newName;
 
 -- Editar Email
 public setEmail: seq of char ==> ()
(*@
\label{setPais:211}
@*)
 setEmail(newEmail) == email := newEmail
 pre newEmail <> undefined
 post email = newEmail;
 
 -- Editar Telefone
 public setTelefone: nat1 ==> ()
(*@
\label{setLocalidade:217}
@*)
 setTelefone(newTelefone) == telefone := newTelefone
 pre newTelefone <> undefined
 post telefone = newTelefone;

 -- Editar Pais
 public setPais: seq of char ==> ()
(*@
\label{setEscolaridade:223}
@*)
 setPais(newPais) == pais := newPais
 pre newPais <> undefined
 post pais = newPais;
 
 -- Editar Localidade
 public setLocalidade: seq of char ==> ()
 setLocalidade(newLocalidade) == localidade := newLocalidade
 pre newLocalidade <> undefined
 post localidade = newLocalidade;
 
(*@
\label{DaysOfMonth:233}
@*)
 -- Editar Escolaridade
 public setEscolaridade: Escolaridade ==> ()
 setEscolaridade(newEscolaridade) == escolaridade := newEscolaridade
 pre newEscolaridade <> undefined
 post escolaridade = newEscolaridade;
 
functions
-- TODO Define functiones here

-- Retorna o número de dias do mês num dado ano
   public static DaysOfMonth(month,year : nat1) r : nat1 == (
    if month = 1 or month = 3 or month = 5 or month = 7 or month = 8 or month = 10 or month = 12 then
     31
    else if month = 2 and ((year mod 4 = 0 and year mod 100 <> 0) or year mod 400 = 0) then
     29
    else if month = 2 then
     28
    else
     30
   )

traces
-- TODO Define Combinatorial Test Traces here

end Utilizador
\end{vdmpp}
\bigskip
\begin{longtable}{|l|r|r|r|}
\hline
Function or operation & Line & Coverage & Calls \\
\hline
\hline
\hyperref[DaysOfMonth:233]{DaysOfMonth} & 233&100.0\% & 1 \\
\hline
\hyperref[Utilizador:32]{Utilizador} & 32&100.0\% & 5 \\
\hline
\hyperref[addInteresse:92]{addInteresse} & 92&100.0\% & 7 \\
\hline
\hyperref[addSkills:111]{addSkills} & 111&100.0\% & 7 \\
\hline
\hyperref[getDataNasccimento:197]{getDataNasccimento} & 197&100.0\% & 1 \\
\hline
\hyperref[getEmail:151]{getEmail} & 151&100.0\% & 14 \\
\hline
\hyperref[getEscolaridade:179]{getEscolaridade} & 179&100.0\% & 4 \\
\hline
\hyperref[getIdade:137]{getIdade} & 137&100.0\% & 1 \\
\hline
\hyperref[getInteresses:104]{getInteresses} & 104&100.0\% & 25 \\
\hline
\hyperref[getLocalidade:172]{getLocalidade} & 172&100.0\% & 2 \\
\hline
\hyperref[getNome:130]{getNome} & 130&100.0\% & 2 \\
\hline
\hyperref[getPais:165]{getPais} & 165&100.0\% & 2 \\
\hline
\hyperref[getSexo:158]{getSexo} & 158&100.0\% & 1 \\
\hline
\hyperref[getSkills:123]{getSkills} & 123&100.0\% & 25 \\
\hline
\hyperref[getTelefone:144]{getTelefone} & 144&100.0\% & 10 \\
\hline
\hyperref[getlistaEscolaridades:186]{getlistaEscolaridades} & 186&100.0\% & 18 \\
\hline
\hyperref[listaEscolaridades:72]{listaEscolaridades} & 72&100.0\% & 1 \\
\hline
\hyperref[removeInteresse:98]{removeInteresse} & 98&100.0\% & 4 \\
\hline
\hyperref[removeSkills:117]{removeSkills} & 117&100.0\% & 2 \\
\hline
\hyperref[setEmail:199]{setEmail} & 199&100.0\% & 1 \\
\hline
\hyperref[setEscolaridade:223]{setEscolaridade} & 223&100.0\% & 1 \\
\hline
\hyperref[setLocalidade:217]{setLocalidade} & 217&100.0\% & 1 \\
\hline
\hyperref[setNome:193]{setNome} & 193&100.0\% & 2 \\
\hline
\hyperref[setPais:211]{setPais} & 211&100.0\% & 1 \\
\hline
\hyperref[setTelefone:205]{setTelefone} & 205&100.0\% & 1 \\
\hline
\hyperref[verificarEscolaridade:57]{verificarEscolaridade} & 57&100.0\% & 3 \\
\hline
\hline
Utilizador.vdmpp & & 100.0\% & 142 \\
\hline
\end{longtable}

