\begin{vdmpp}[breaklines=true]
class Utilizador
types
-- TODO Define types here
 public String = seq of char;
 public Sexo = <Masculino> | <Feminino>;
 public Escolaridade = <Secundario> | <Licenciatura> | <Mestrado> | <Doutoramento>;
 public Interesses = set of String;
 public Skills = set of String;
 
values
-- TODO Define values here

instance variables
-- TODO Define instance variables here
 private nome: String;
 private email: String;
 private sexo: Sexo;
 private escolaridade: Escolaridade;
 private idade: nat1;
 private telefone: nat1;
 private pais: String;
(*@
\label{Utilizador:22}
@*)
 private localidade: String;
 private interesses: Interesses := {};
 private skills: Skills := {};
 
operations
-- TODO Define operations here

 --Construtor
 public Utilizador: String * String * Sexo * Escolaridade * String * String * nat1 * nat1 ==> Utilizador
 Utilizador(nm,emailC,sexoC,escolaridadeC,paisC,localidadeC,idadeC,telefoneC) == (
  nome := nm;
  email := emailC;
  sexo := sexoC;
(*@
\label{addInteresse:35}
@*)
  escolaridade := escolaridadeC;
  idade := idadeC;
  telefone := telefoneC;
  pais := paisC;
  localidade := localidadeC;
  return self
(*@
\label{removeInteresse:41}
@*)
 )
 post interesses = {} and
   skills = {} and
   nome = nm and
   email = emailC and
   sexo = sexoC and
(*@
\label{getInteresses:47}
@*)
   idade = idadeC and
   telefone = telefoneC and
   pais = paisC and
   localidade = localidadeC;
 
 -- Adicionar interesses
 public addInteresse: String ==> ()
(*@
\label{getNome:54}
@*)
 addInteresse(String) == interesses := interesses union {String}
 pre String not in set interesses
 post interesses = interesses~ union {String};
 
 -- Remover interesses
 public removeInteresse: String ==> ()
 removeInteresse(String) == interesses := interesses \ {String}
 pre String in set interesses
 post interesses = interesses~ \ {String};
 
 -- Retorna os interesses
 public pure getInteresses : () ==> Interesses
 getInteresses() ==
 (
  return interesses;
(*@
\label{addSkills:69}
@*)
 );
 
 -- Adicionar skills
 public addSkills: String ==> ()
 addSkills(String) == skills := skills union {String}
 pre String not in set skills
(*@
\label{removeSkills:75}
@*)
 post skills = skills~ union {String};
 
 -- Remover skills
 public removeSkills: String ==> ()
 removeSkills(String) == skills := skills \ {String}
 pre String in set skills
(*@
\label{getSkills:81}
@*)
 post skills = skills~ \ {String};
 
 -- Retorna as skills
 public pure getSkills : () ==> Skills
 getSkills() ==
 (
  return skills;
 );
 
 -- Retorna o nome
 public pure getNome : () ==> String
 getNome() ==
 (
  return nome;
(*@
\label{getIdade:95}
@*)
 );
 
 -- Retorna a idade
 public pure getIdade : () ==> nat1
 getIdade() ==
 (
  return idade;
(*@
\label{getTelefone:102}
@*)
 );
 
 -- Retorna o telefone
 public pure getTelefone : () ==> nat1
 getTelefone() ==
 (
  return telefone;
(*@
\label{getEmail:109}
@*)
 );
 
 -- Retorna o email
 public pure getEmail : () ==> String
 getEmail() ==
 (
  return email;
(*@
\label{getSexo:116}
@*)
 );
 
 -- Retorna o sexo
 public pure getSexo : () ==> Sexo
 getSexo() ==
 (
  return sexo;
(*@
\label{getPais:123}
@*)
 );
 
 -- Retorna o pais
 public pure getPais : () ==> String
 getPais() ==
 (
  return pais;
(*@
\label{getLocalidade:130}
@*)
 );
 
 -- Retorna a localidade
 public pure getLocalidade : () ==> String
 getLocalidade() ==
 (
  return localidade;
(*@
\label{setNome:137}
@*)
 );
 
 -- Retorna a Escolaridade
(*@
\label{getEscolaridade:140}
@*)
 public pure getEscolaridade : () ==> Escolaridade
 getEscolaridade() ==
 (
(*@
\label{setEmail:143}
@*)
  return escolaridade;
 );
 
 -- Editar Nome
 public setNome: String ==> ()
 setNome(newName) == nome := newName
(*@
\label{setTelefone:149}
@*)
 pre newName <> undefined
 post nome = newName;
 
 -- Editar Email
 public setEmail: String ==> ()
 setEmail(newEmail) == email := newEmail
(*@
\label{setParis:155}
@*)
(*@
\label{setPais:155}
@*)
 pre newEmail <> undefined
 post email = newEmail;
 
 -- Editar Telefone
 public setTelefone: nat1 ==> ()
 setTelefone(newTelefone) == telefone := newTelefone
(*@
\label{setLocalidade:161}
@*)
 pre newTelefone <> undefined
 post telefone = newTelefone;

 -- Editar Pais
 public setPais: String ==> ()
 setPais(newPais) == pais := newPais
(*@
\label{setEscolaridade:167}
@*)
 pre newPais <> undefined
 post pais = newPais;
 
 -- Editar Localidade
 public setLocalidade: String ==> ()
 setLocalidade(newLocalidade) == localidade := newLocalidade
 pre newLocalidade <> undefined
 post localidade = newLocalidade;
 
 -- Editar Escolaridade
 public setEscolaridade: Escolaridade ==> ()
 setEscolaridade(newEscolaridade) == escolaridade := newEscolaridade
 pre newEscolaridade <> undefined
 post escolaridade = newEscolaridade;
 
functions
-- TODO Define functiones here

traces
-- TODO Define Combinatorial Test Traces here

end Utilizador
\end{vdmpp}
\bigskip
\begin{longtable}{|l|r|r|r|}
\hline
Function or operation & Line & Coverage & Calls \\
\hline
\hline
\hyperref[Utilizador:22]{Utilizador} & 22&100.0\% & 8 \\
\hline
\hyperref[addInteresse:35]{addInteresse} & 35&100.0\% & 12 \\
\hline
\hyperref[addSkills:69]{addSkills} & 69&100.0\% & 12 \\
\hline
\hyperref[getEmail:109]{getEmail} & 109&100.0\% & 12 \\
\hline
\hyperref[getEscolaridade:140]{getEscolaridade} & 140&100.0\% & 11 \\
\hline
\hyperref[getIdade:95]{getIdade} & 95&100.0\% & 6 \\
\hline
\hyperref[getInteresses:47]{getInteresses} & 47&100.0\% & 12 \\
\hline
\hyperref[getLocalidade:130]{getLocalidade} & 130&100.0\% & 12 \\
\hline
\hyperref[getNome:54]{getNome} & 54&100.0\% & 12 \\
\hline
\hyperref[getPais:123]{getPais} & 123&100.0\% & 12 \\
\hline
\hyperref[getSexo:116]{getSexo} & 116&100.0\% & 6 \\
\hline
\hyperref[getSkills:81]{getSkills} & 81&100.0\% & 12 \\
\hline
\hyperref[getTelefone:102]{getTelefone} & 102&100.0\% & 12 \\
\hline
\hyperref[removeInteresse:41]{removeInteresse} & 41&100.0\% & 6 \\
\hline
\hyperref[removeSkills:75]{removeSkills} & 75&100.0\% & 6 \\
\hline
\hyperref[setEmail:143]{setEmail} & 143&100.0\% & 6 \\
\hline
\hyperref[setEscolaridade:167]{setEscolaridade} & 167&100.0\% & 5 \\
\hline
\hyperref[setLocalidade:161]{setLocalidade} & 161&100.0\% & 6 \\
\hline
\hyperref[setNome:137]{setNome} & 137&100.0\% & 6 \\
\hline
\hyperref[setPais:155]{setPais} & 155&100.0\% & 6 \\
\hline
\hyperref[setParis:155]{setParis} & 155&100.0\% & 6 \\
\hline
\hyperref[setTelefone:149]{setTelefone} & 149&100.0\% & 6 \\
\hline
\hline
Utilizador.vdmpp & & 100.0\% & 192 \\
\hline
\end{longtable}

