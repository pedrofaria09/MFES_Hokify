\begin{vdmpp}[breaklines=true]
class Utilizador
types
-- TODO Define types here
 public String = seq of char;
 public Sexo = <Masculino> | <Feminino>;
 public Escolaridade = <Secundario> | <Licenciatura> | <Mestrado> | <Doutoramento>;
 public ListaEscolaridade = set of Escolaridade;
 public Interesses = set of String;
 public Skills = set of String;
 
values
-- TODO Define values here

instance variables
-- TODO Define instance variables here
 private nome: seq of char;
 private email: seq of char;
 private sexo: Sexo;
 private escolaridade: Escolaridade;
 private idade: nat1;
 private telefone: nat1;
 private pais: seq of char;
 private localidade: seq of char;
 private interesses: set of String := {};
 private skills: set of String := {};
 private listaEscolaridade: ListaEscolaridade := {};
 
operations
-- TODO Define operations here

 --Construtor
(*@
\label{Utilizador:32}
@*)
 public Utilizador: seq of char * seq of char * Sexo * Escolaridade * seq of char * seq of char * nat1 * nat1 ==> Utilizador
 Utilizador(nm,emailC,sexoC,escolaridadeC,paisC,localidadeC,idadeC,telefoneC) == (
  nome := nm;
  email := emailC;
  sexo := sexoC;
  escolaridade := escolaridadeC;
  listaEscolaridades(escolaridade);
  idade := idadeC;
  telefone := telefoneC;
  pais := paisC;
  localidade := localidadeC;
  return self
 )
 pre verificarEscolaridade(escolaridadeC, idadeC)
 post interesses = {} and
   skills = {} and
   nome = nm and
   email = emailC and
   sexo = sexoC and
   idade = idadeC and
   telefone = telefoneC and
   pais = paisC and
   localidade = localidadeC;
(*@
\label{listaEscolaridades:55}
@*)
 
(*@
\label{verificarEscolaridade:56}
@*)
 public pure verificarEscolaridade: Escolaridade * nat1 ==> bool
 verificarEscolaridade(escola, idd)==(
 if (escola = <Secundario> and idd > 17) then(
  return true;
 )elseif(escola = <Licenciatura> and idd > 20) then(
  return true;
 )elseif(escola = <Mestrado> and idd > 22) then(
  return true;
 )elseif(escola = <Doutoramento> and idd > 24) then(
  return true;
 )else
  return false;
 )pre escolaridade <> undefined;
 
 public listaEscolaridades: Escolaridade ==> ()
 listaEscolaridades(tipo)==(
 if tipo = <Doutoramento> then (
  listaEscolaridade := listaEscolaridade union {<Secundario>};
  listaEscolaridade := listaEscolaridade union {<Licenciatura>};
(*@
\label{addInteresse:75}
@*)
  listaEscolaridade := listaEscolaridade union {<Mestrado>};
  listaEscolaridade := listaEscolaridade union {<Doutoramento>};
 )elseif tipo = <Mestrado> then(
  listaEscolaridade := listaEscolaridade union {<Secundario>};
  listaEscolaridade := listaEscolaridade union {<Licenciatura>};
  listaEscolaridade := listaEscolaridade union {<Mestrado>};
(*@
\label{removeInteresse:81}
@*)
 )elseif tipo = <Licenciatura> then(
  listaEscolaridade := listaEscolaridade union {<Secundario>};
  listaEscolaridade := listaEscolaridade union {<Licenciatura>};
 )else(
  listaEscolaridade := listaEscolaridade union {<Secundario>};
 );
(*@
\label{getInteresses:87}
@*)
 )pre tipo <> undefined;
 
 -- Adicionar interesses
 public addInteresse: seq of char ==> ()
 addInteresse(String) == interesses := interesses union {String}
 pre String not in set interesses
 post interesses = interesses~ union {String};
(*@
\label{addSkills:94}
@*)
 
 -- Remover interesses
 public removeInteresse: seq of char ==> ()
 removeInteresse(String) == interesses := interesses \ {String}
 pre String in set interesses
 post interesses = interesses~ \ {String};
(*@
\label{removeSkills:100}
@*)
 
 -- Retorna os interesses
 public pure getInteresses : () ==> set of String
 getInteresses() ==
 (
  return interesses;
(*@
\label{getSkills:106}
@*)
 );
 
 -- Adicionar skills
 public addSkills: seq of char ==> ()
 addSkills(String) == skills := skills union {String}
 pre String not in set skills
 post skills = skills~ union {String};
(*@
\label{getNome:113}
@*)
 
 -- Remover skills
 public removeSkills: seq of char ==> ()
 removeSkills(String) == skills := skills \ {String}
 pre String in set skills
 post skills = skills~ \ {String};
 
(*@
\label{getIdade:120}
@*)
 -- Retorna as skills
 public pure getSkills : () ==> set of String
 getSkills() ==
 (
  return skills;
 );
 
(*@
\label{getTelefone:127}
@*)
 -- Retorna o nome
 public pure getNome : () ==> seq of char
 getNome() ==
 (
  return nome;
 );
 
(*@
\label{getEmail:134}
@*)
 -- Retorna a idade
 public pure getIdade : () ==> nat1
 getIdade() ==
 (
  return idade;
 );
 
(*@
\label{getSexo:141}
@*)
 -- Retorna o telefone
 public pure getTelefone : () ==> nat1
 getTelefone() ==
 (
  return telefone;
 );
 
(*@
\label{getPais:148}
@*)
 -- Retorna o email
 public pure getEmail : () ==> seq of char
 getEmail() ==
 (
  return email;
 );
 
(*@
\label{getLocalidade:155}
@*)
 -- Retorna o sexo
 public pure getSexo : () ==> Sexo
 getSexo() ==
 (
  return sexo;
 );
 
(*@
\label{getEscolaridade:162}
@*)
 -- Retorna o pais
 public pure getPais : () ==> seq of char
 getPais() ==
 (
  return pais;
 );
 
(*@
\label{getlistaEscolaridades:169}
@*)
 -- Retorna a localidade
 public pure getLocalidade : () ==> seq of char
 getLocalidade() ==
 (
  return localidade;
 );
 
(*@
\label{setNome:176}
@*)
 -- Retorna a escolaridade
 public pure getEscolaridade : () ==> Escolaridade
 getEscolaridade() ==
 (
  return escolaridade;
 );
(*@
\label{setEmail:182}
@*)
 
 -- Retorna o listaEscolaridades
 public pure getlistaEscolaridades : () ==> ListaEscolaridade
 getlistaEscolaridades() ==
 (
  return listaEscolaridade;
(*@
\label{setTelefone:188}
@*)
 );
 
 -- Editar Nome
 public setNome: seq of char ==> ()
 setNome(newName) == nome := newName
 pre newName <> undefined
(*@
\label{setPais:194}
@*)
 post nome = newName;
 
 -- Editar Email
 public setEmail: seq of char ==> ()
 setEmail(newEmail) == email := newEmail
 pre newEmail <> undefined
(*@
\label{setLocalidade:200}
@*)
 post email = newEmail;
 
 -- Editar Telefone
 public setTelefone: nat1 ==> ()
 setTelefone(newTelefone) == telefone := newTelefone
 pre newTelefone <> undefined
(*@
\label{setEscolaridade:206}
@*)
 post telefone = newTelefone;

 -- Editar Pais
 public setPais: seq of char ==> ()
 setPais(newPais) == pais := newPais
 pre newPais <> undefined
 post pais = newPais;
 
 -- Editar Localidade
 public setLocalidade: seq of char ==> ()
 setLocalidade(newLocalidade) == localidade := newLocalidade
 pre newLocalidade <> undefined
 post localidade = newLocalidade;
 
 -- Editar Escolaridade
 public setEscolaridade: Escolaridade ==> ()
 setEscolaridade(newEscolaridade) == escolaridade := newEscolaridade
 pre newEscolaridade <> undefined
 post escolaridade = newEscolaridade;
 
functions
-- TODO Define functiones here

traces
-- TODO Define Combinatorial Test Traces here

end Utilizador
\end{vdmpp}
\bigskip
\begin{longtable}{|l|r|r|r|}
\hline
Function or operation & Line & Coverage & Calls \\
\hline
\hline
\hyperref[Utilizador:32]{Utilizador} & 32&100.0\% & 20 \\
\hline
\hyperref[addInteresse:75]{addInteresse} & 75&100.0\% & 56 \\
\hline
\hyperref[addSkills:94]{addSkills} & 94&100.0\% & 56 \\
\hline
\hyperref[getEmail:134]{getEmail} & 134&100.0\% & 44 \\
\hline
\hyperref[getEscolaridade:162]{getEscolaridade} & 162&100.0\% & 17 \\
\hline
\hyperref[getIdade:120]{getIdade} & 120&100.0\% & 4 \\
\hline
\hyperref[getInteresses:87]{getInteresses} & 87&100.0\% & 100 \\
\hline
\hyperref[getLocalidade:155]{getLocalidade} & 155&100.0\% & 8 \\
\hline
\hyperref[getNome:113]{getNome} & 113&100.0\% & 8 \\
\hline
\hyperref[getPais:148]{getPais} & 148&100.0\% & 8 \\
\hline
\hyperref[getSexo:141]{getSexo} & 141&100.0\% & 4 \\
\hline
\hyperref[getSkills:106]{getSkills} & 106&100.0\% & 100 \\
\hline
\hyperref[getTelefone:127]{getTelefone} & 127&100.0\% & 34 \\
\hline
\hyperref[getlistaEscolaridades:169]{getlistaEscolaridades} & 169&100.0\% & 72 \\
\hline
\hyperref[listaEscolaridades:55]{listaEscolaridades} & 55&100.0\% & 4 \\
\hline
\hyperref[removeInteresse:81]{removeInteresse} & 81&100.0\% & 8 \\
\hline
\hyperref[removeSkills:100]{removeSkills} & 100&100.0\% & 8 \\
\hline
\hyperref[setEmail:182]{setEmail} & 182&100.0\% & 4 \\
\hline
\hyperref[setEscolaridade:206]{setEscolaridade} & 206&100.0\% & 5 \\
\hline
\hyperref[setLocalidade:200]{setLocalidade} & 200&100.0\% & 4 \\
\hline
\hyperref[setNome:176]{setNome} & 176&100.0\% & 8 \\
\hline
\hyperref[setPais:194]{setPais} & 194&100.0\% & 4 \\
\hline
\hyperref[setTelefone:188]{setTelefone} & 188&100.0\% & 4 \\
\hline
\hyperref[verificarEscolaridade:56]{verificarEscolaridade} & 56&100.0\% & 22 \\
\hline
\hline
Utilizador.vdmpp & & 100.0\% & 602 \\
\hline
\end{longtable}

