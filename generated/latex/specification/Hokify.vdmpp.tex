\begin{vdmpp}[breaklines=true]
class Hokify
types
-- TODO Define types here
 public String = seq of char;
 public Utilizadores = set of Utilizador;
 public Trabalhos = set of Trabalho;
 public Escolaridade = <Secundario> | <Licenciatura> | <Mestrado> | <Doutoramento>;
 
values
-- TODO Define values here
instance variables
-- TODO Define instance variables here
 private utilizadores: set of Utilizador := {};
 private trabalhos: set of Trabalho := {};
 
operations
-- TODO Define operations here
 
 --Construtor
(*@
\label{Hokify:20}
@*)
 public Hokify: () ==> Hokify
 Hokify()==(return self)
 post utilizadores = {} and
   trabalhos = {};
 
 
 --Adicionar Utilizadores
(*@
\label{addUtilizadores:27}
@*)
 public addUtilizadores: Utilizador ==> ()
 addUtilizadores(utilizador) == utilizadores := utilizadores union {utilizador}
 pre sameUser(utilizador)
 post utilizadores = utilizadores~ union {utilizador};
 
 -- Remover Utilizadores
(*@
\label{removeUtilizadores:33}
@*)
 public removeUtilizadores: Utilizador ==> ()
 removeUtilizadores(utilizador) == utilizadores := utilizadores \ {utilizador}
 pre not sameUser(utilizador)
 post utilizadores = utilizadores~ \ {utilizador};
 
 --Adicionar Trabalhos
(*@
\label{addTrabalhos:39}
@*)
 public addTrabalhos: Trabalho ==> ()
 addTrabalhos(trabalho) == trabalhos := trabalhos union {trabalho}
 pre sameTrabalho(trabalho)
 post trabalhos = trabalhos~ union {trabalho};
 
 -- Remover Trabalhos
(*@
\label{removeTrabalhos:45}
@*)
 public removeTrabalhos: Trabalho ==> ()
 removeTrabalhos(trabalho) == trabalhos := trabalhos \ {trabalho}
 pre not sameTrabalho(trabalho)
 post trabalhos = trabalhos~ \ {trabalho};
 
 -- Retorna os utilizadores
(*@
\label{getUtilizadores:51}
@*)
 public pure getUtilizadores : () ==> set of Utilizador
 getUtilizadores() ==
 (
  return utilizadores;
 );
 
 -- Retorna os trabalhos
(*@
\label{getTrabalhos:58}
@*)
 public pure getTrabalhos : () ==> set of Trabalho
 getTrabalhos() ==
 (
  return trabalhos;
 );
 
 -- Retorna os trabalhos por nome
(*@
\label{getTrabalhosPorNome:65}
@*)
 public pure getTrabalhosPorNome: seq of char ==> set of Trabalho
 getTrabalhosPorNome(nome) == (return {trabalhos | trabalhos in set trabalhos & trabalhos.nomeSemelhante(nome)})
 pre len nome > 0;
 
 --Retorna os trabalhos por interesses
(*@
\label{getTrabalhosPorInteresses:70}
@*)
 public pure getTrabalhosPorInteresses: seq of char ==> set of Trabalho
 getTrabalhosPorInteresses(nome) == (
  dcl results: set of Trabalho := {};
  for all tr in set trabalhos do
   if nome in set tr.getInteresses() then
    results := results union {tr};
  return results;
 )
 pre len nome > 0;
 
 --Retorna os trabalhos por skills
(*@
\label{getTrabalhosPorSkills:81}
@*)
 public pure getTrabalhosPorSkills: seq of char ==> set of Trabalho
 getTrabalhosPorSkills(nome) == (
  dcl results: set of Trabalho := {};
  for all tr in set trabalhos do
   if nome in set tr.getSkills() then
    results := results union {tr};
  return results;
 )
 pre len nome > 0;

 --Retorna os trabalhos por Escolaridade
(*@
\label{getTrabalhosPorEscolaridade:92}
@*)
 public pure getTrabalhosPorEscolaridade: Escolaridade ==> set of Trabalho
 getTrabalhosPorEscolaridade(nome) == (
  dcl results: set of Trabalho := {};
  for all tr in set trabalhos do
   if nome in set tr.getlistaEscolaridades() then
    results := results union {tr};
  return results;
 )
 pre nome <> undefined;
 
 --Retorna os trabalhos por Utilizador (Escolaridade,Skills,interesses)
(*@
\label{getTrabalhosPorUtilizador:103}
@*)
 public pure getTrabalhosPorUtilizador: Utilizador ==> set of Trabalho
 getTrabalhosPorUtilizador(usr) == (
  dcl results_escolaridade: set of Trabalho := {};
  dcl results_skills: set of Trabalho := {};
  dcl results_interesses: set of Trabalho := {};
  dcl trabalhos_temp: set of Trabalho := {};
  results_escolaridade := getTrabalhosPorEscolaridade(usr.getEscolaridade());
  
  for all skill in set usr.getSkills() do
   trabalhos_temp := getTrabalhosPorSkills(skill);
   for all skill_temp in set trabalhos_temp do
    if skill_temp not in set results_skills then
     results_skills := results_skills union {skill_temp};
     
  for all interesse in set usr.getInteresses() do
   trabalhos_temp := getTrabalhosPorInteresses(interesse);
   for all interesse_temp in set trabalhos_temp do
    if interesse_temp not in set results_interesses then
     results_interesses := results_interesses union {interesse_temp};
  return (results_escolaridade inter results_skills inter results_interesses);
 )
 pre usr <> undefined;
 
 --Retorna os trabalhos por pais
(*@
\label{getTrabalhosPorPais:127}
@*)
 public pure getTrabalhosPorPais: seq of char ==> set of Trabalho
 getTrabalhosPorPais(nome) == (
  dcl results: set of Trabalho := {};
  for all tr in set trabalhos do
   if nome = tr.getPais() then
    results := results union {tr};
  return results;
 )
 pre len nome > 0;
 
 --Retorna os trabalhos por localidade
(*@
\label{getTrabalhosPorLocalidade:138}
@*)
 public pure getTrabalhosPorLocalidade: seq of char ==> set of Trabalho
 getTrabalhosPorLocalidade(nome) == (
  dcl results: set of Trabalho := {};
  for all tr in set trabalhos do
   if nome = tr.getLocalidade() then
    results := results union {tr};
  return results;
 )
 pre len nome > 0;
 
 --Retorna os utilizadres por interesses
(*@
\label{getUtilizadoresPorInteresses:149}
@*)
 public pure getUtilizadoresPorInteresses: seq of char ==> set of Utilizador
 getUtilizadoresPorInteresses(nome) == (
  dcl results: set of Utilizador := {};
  for all tr in set utilizadores do
   if nome in set tr.getInteresses() then
    results := results union {tr};
  return results;
 )
 pre len nome > 0;
 
 --Retorna os utilizadres por skills
(*@
\label{getUtilizadoresPorSkills:160}
@*)
 public pure getUtilizadoresPorSkills: seq of char ==> set of Utilizador
 getUtilizadoresPorSkills(nome) == (
  dcl results: set of Utilizador := {};
  for all tr in set utilizadores do
   if nome in set tr.getSkills() then
    results := results union {tr};
  return results;
 )
 pre len nome > 0;

 --Retorna os utilizadres por Escolaridade
(*@
\label{getUtilizadoresPorEscolaridade:171}
@*)
 public pure getUtilizadoresPorEscolaridade: Escolaridade ==> set of Utilizador
 getUtilizadoresPorEscolaridade(nome) == (
  dcl results: set of Utilizador := {};
  for all tr in set utilizadores do
   if nome in set tr.getlistaEscolaridades() then
    results := results union {tr};
  return results;
 )
 pre nome <> undefined;
 
 --Retorna os utilizadores por Trabalhos (Escolaridade,Skills,interesses)
 public pure getTrabalhosPorUtilizador: Trabalho ==> set of Utilizador
 getTrabalhosPorUtilizador(trab) == (
  dcl results_escolaridade: set of Utilizador := {};
  dcl results_skills: set of Utilizador := {};
  dcl results_interesses: set of Utilizador := {};
  dcl utilizadores_temp: set of Utilizador := {};
  results_escolaridade := getUtilizadoresPorEscolaridade(trab.getEscolaridade());
  
  for all skill in set trab.getSkills() do
   utilizadores_temp := getUtilizadoresPorSkills(skill);
   for all skill_temp in set utilizadores_temp do
    if skill_temp not in set results_skills then
     results_skills := results_skills union {skill_temp};
     
  for all interesse in set trab.getInteresses() do
   utilizadores_temp := getUtilizadoresPorInteresses(interesse);
   for all interesse_temp in set utilizadores_temp do
    if interesse_temp not in set results_interesses then
     results_interesses := results_interesses union {interesse_temp};
     
  return (results_escolaridade inter results_skills inter results_interesses);
 )
 pre trab <> undefined;
 
 -- Verifica se o utilizador existe por email ou telefone
(*@
\label{sameUser:207}
@*)
 public pure sameUser: Utilizador ==> bool
 sameUser(user) ==(
  for all u in set utilizadores do
   if (u.getEmail() = user.getEmail() or 
     u.getTelefone() = user.getTelefone()) then
    return false;
  return true;
 )
 pre user <> undefined;
 
 -- Verifica se o trabalho existe por email ou nome ou entidade
(*@
\label{sameTrabalho:218}
@*)
 public pure sameTrabalho: Trabalho ==> bool
 sameTrabalho(trab) ==(
  for all u in set trabalhos do
   if (u.getEmail() = trab.getEmail() or 
     u.getNome() = trab.getNome() or 
     u.getEntidade() = trab.getEntidade()) then
    return false;
  return true;
 )
 pre trab <> undefined;
 
functions
-- TODO Define functiones here
traces
-- TODO Define Combinatorial Test Traces here
end Hokify
\end{vdmpp}
\bigskip
\begin{longtable}{|l|r|r|r|}
\hline
Function or operation & Line & Coverage & Calls \\
\hline
\hline
\hyperref[Hokify:20]{Hokify} & 20&100.0\% & 1 \\
\hline
\hyperref[addTrabalhos:39]{addTrabalhos} & 39&100.0\% & 3 \\
\hline
\hyperref[addUtilizadores:27]{addUtilizadores} & 27&100.0\% & 3 \\
\hline
\hyperref[getTrabalhos:58]{getTrabalhos} & 58&100.0\% & 4 \\
\hline
\hyperref[getTrabalhosPorEscolaridade:92]{getTrabalhosPorEscolaridade} & 92&100.0\% & 6 \\
\hline
\hyperref[getTrabalhosPorInteresses:70]{getTrabalhosPorInteresses} & 70&100.0\% & 7 \\
\hline
\hyperref[getTrabalhosPorLocalidade:138]{getTrabalhosPorLocalidade} & 138&100.0\% & 2 \\
\hline
\hyperref[getTrabalhosPorNome:65]{getTrabalhosPorNome} & 65&100.0\% & 3 \\
\hline
\hyperref[getTrabalhosPorPais:127]{getTrabalhosPorPais} & 127&100.0\% & 3 \\
\hline
\hyperref[getTrabalhosPorSkills:81]{getTrabalhosPorSkills} & 81&100.0\% & 7 \\
\hline
\hyperref[getTrabalhosPorUtilizador:103]{getTrabalhosPorUtilizador} & 103&100.0\% & 2 \\
\hline
\hyperref[getUtilizadores:51]{getUtilizadores} & 51&100.0\% & 4 \\
\hline
\hyperref[getUtilizadoresPorEscolaridade:171]{getUtilizadoresPorEscolaridade} & 171&100.0\% & 6 \\
\hline
\hyperref[getUtilizadoresPorInteresses:149]{getUtilizadoresPorInteresses} & 149&100.0\% & 7 \\
\hline
\hyperref[getUtilizadoresPorSkills:160]{getUtilizadoresPorSkills} & 160&100.0\% & 7 \\
\hline
\hyperref[removeTrabalhos:45]{removeTrabalhos} & 45&100.0\% & 1 \\
\hline
\hyperref[removeUtilizadores:33]{removeUtilizadores} & 33&100.0\% & 1 \\
\hline
\hyperref[sameTrabalho:218]{sameTrabalho} & 218&100.0\% & 5 \\
\hline
\hyperref[sameUser:207]{sameUser} & 207&100.0\% & 2 \\
\hline
\hline
Hokify.vdmpp & & 100.0\% & 74 \\
\hline
\end{longtable}

