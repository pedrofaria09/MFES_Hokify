\begin{vdmpp}[breaklines=true]
class Hokify
types
-- TODO Define types here
 public String = seq of char;
 public Utilizadores = set of Utilizador;
 public Trabalhos = set of Trabalho;
 public Escolaridade = <Secundario> | <Licenciatura> | <Mestrado> | <Doutoramento>;
 
values
-- TODO Define values here
instance variables
-- TODO Define instance variables here
 private utilizadores: Utilizadores := {};
 private trabalhos: Trabalhos := {};
 
operations
-- TODO Define operations here
(*@
\label{Hokify:18}
@*)
 
 --Construtor
 public Hokify: () ==> Hokify
 Hokify()==(return self);
(*@
\label{addUtilizadores:22}
@*)
(*@
\label{addSkills:22}
@*)
 
 
 --Adicionar Utilizadores
 public addUtilizadores: Utilizador ==> ()
(*@
\label{mapEscolaridades:26}
@*)
 addUtilizadores(utilizador) == utilizadores := utilizadores union {utilizador}
 pre sameUser(utilizador)
(*@
\label{addTrabalhos:28}
@*)
 post utilizadores = utilizadores~ union {utilizador};
 
 --Adicionar Trabalhos
 public addTrabalhos: Trabalho ==> ()
 addTrabalhos(trabalho) == trabalhos := trabalhos union {trabalho}
(*@
\label{sameUser:33}
@*)
 pre sameTrabalho(trabalho)
(*@
\label{getSkills:34}
@*)
(*@
\label{getUtilizadores:34}
@*)
 post trabalhos = trabalhos~ union {trabalho};
 
 -- Retorna os utilizadores
 public pure getUtilizadores : () ==> Utilizadores
 getUtilizadores() ==
 (
  return utilizadores;
(*@
\label{sameTrabalho:41}
@*)
(*@
\label{getTrabalhos:41}
@*)
 );
 
 -- Retorna os trabalhos
 public pure getTrabalhos : () ==> Trabalhos
 getTrabalhos() ==
 (
  return trabalhos;
 );
 
 -- Retorna os trabalhos por nome
 public pure getTrabalhosPorNome: String ==> Trabalhos
 getTrabalhosPorNome(nome) == (return {trabalhos | trabalhos in set trabalhos & trabalhos.nomeSemelhante(nome)})
 pre len nome > 0;
(*@
\label{getTrabalhosPorInteresses:54}
@*)
 
 --Retorna os trabalhos por interesses
 public pure getTrabalhosPorInteresses: String ==> Trabalhos
 getTrabalhosPorInteresses(nome) == (
  dcl results: Trabalhos := {};
(*@
\label{searchByName:59}
@*)
(*@
\label{getTrabalhosPorNome:59}
@*)
  for all tr in set trabalhos do
   if nome in set tr.getInteresses() then
    results := results union {tr};
  return results;
 );
(*@
\label{getTrabalhosPorSkills:64}
@*)
 
 --Retorna os trabalhos por skills
 public pure getTrabalhosPorSkills: String ==> Trabalhos
 getTrabalhosPorSkills(nome) == (
  dcl results: Trabalhos := {};
  for all tr in set trabalhos do
   if nome in set tr.getSkills() then
    results := results union {tr};
  return results;
 );
(*@
\label{getTrabalhosPorLocalidade:74}
@*)
(*@
\label{getTrabalhosPorPais:74}
@*)

 --Retorna os trabalhos por Escolaridade
(*@
\label{getTrabalhosPorEscolaridade:76}
@*)
 public pure getTrabalhosPorEscolaridade: Escolaridade ==> Trabalhos
 getTrabalhosPorEscolaridade(nome) == (
  dcl results: Trabalhos := {};
  for all tr in set trabalhos do
   if nome in set tr.getlistaEscolaridades() then
    results := results union {tr};
  return results;
 );
 
 --Retorna os trabalhos por Utilizador (Escolaridade,Skills,interesses)
(*@
\label{getTrabalhosPorUtilizador:86}
@*)
 public pure getTrabalhosPorUtilizador: Utilizador ==> Trabalhos
 getTrabalhosPorUtilizador(usr) == (
  dcl results_escolaridade: Trabalhos := {};
  dcl results_skills: Trabalhos := {};
  dcl results_interesses: Trabalhos := {};
  dcl trabalhos_temp: Trabalhos := {};
  results_escolaridade := getTrabalhosPorEscolaridade(usr.getEscolaridade());
  
  for all skill in set usr.getSkills() do
   trabalhos_temp := getTrabalhosPorSkills(skill);
   for all skill_temp in set trabalhos_temp do
    if skill_temp not in set results_skills then
     results_skills := results_skills union {skill_temp};
     
  for all interesse in set usr.getInteresses() do
   trabalhos_temp := getTrabalhosPorInteresses(interesse);
   for all interesse_temp in set trabalhos_temp do
    if interesse_temp not in set results_interesses then
     results_interesses := results_interesses union {interesse_temp};
  return (results_escolaridade inter results_skills inter results_interesses);
 );
 
 --Retorna os trabalhos por pais
 public pure getTrabalhosPorPais: String ==> Trabalhos
 getTrabalhosPorPais(nome) == (
  dcl results: Trabalhos := {};
  for all tr in set trabalhos do
   if nome = tr.getPais() then
    results := results union {tr};
  return results;
 );
 
 --Retorna os trabalhos por localidade
 public pure getTrabalhosPorLocalidade: String ==> Trabalhos
 getTrabalhosPorLocalidade(nome) == (
  dcl results: Trabalhos := {};
  for all tr in set trabalhos do
   if nome = tr.getLocalidade() then
    results := results union {tr};
  return results;
 );
(*@
\label{getUtilizadoresPorInteresses:127}
@*)
 
 --Retorna os utilizadres por interesses
 public pure getUtilizadoresPorInteresses: String ==> Utilizadores
 getUtilizadoresPorInteresses(nome) == (
  dcl results: Utilizadores := {};
  for all tr in set utilizadores do
   if nome in set tr.getInteresses() then
    results := results union {tr};
  return results;
 );
(*@
\label{getUtilizadoresPorSkills:137}
@*)
 
 --Retorna os utilizadres por skills
 public pure getUtilizadoresPorSkills: String ==> Utilizadores
 getUtilizadoresPorSkills(nome) == (
  dcl results: Utilizadores := {};
  for all tr in set utilizadores do
   if nome in set tr.getSkills() then
    results := results union {tr};
  return results;
 );
(*@
\label{getUtilizadoresPorEscolaridade:147}
@*)

 --Retorna os utilizadres por Escolaridade
 public pure getUtilizadoresPorEscolaridade: Escolaridade ==> Utilizadores
 getUtilizadoresPorEscolaridade(nome) == (
  dcl results: Utilizadores := {};
  for all tr in set utilizadores do
   if nome in set tr.getlistaEscolaridades() then
    results := results union {tr};
  return results;
 );
 
 --Retorna os utilizadores por Trabalhos (Escolaridade,Skills,interesses)
 public pure getTrabalhosPorUtilizador: Trabalho ==> Utilizadores
 getTrabalhosPorUtilizador(trab) == (
  dcl results_escolaridade: Utilizadores := {};
  dcl results_skills: Utilizadores := {};
  dcl results_interesses: Utilizadores := {};
  dcl utilizadores_temp: Utilizadores := {};
  results_escolaridade := getUtilizadoresPorEscolaridade(trab.getEscolaridade());
  
  for all skill in set trab.getSkills() do
   utilizadores_temp := getUtilizadoresPorSkills(skill);
   for all skill_temp in set utilizadores_temp do
    if skill_temp not in set results_skills then
     results_skills := results_skills union {skill_temp};
     
  for all interesse in set trab.getInteresses() do
   utilizadores_temp := getUtilizadoresPorInteresses(interesse);
   for all interesse_temp in set utilizadores_temp do
    if interesse_temp not in set results_interesses then
     results_interesses := results_interesses union {interesse_temp};
     
  return (results_escolaridade inter results_skills inter results_interesses);
 );
 
 -- Verifica se o utilizador existe por email ou telefone
 public pure sameUser: Utilizador ==> bool
 sameUser(user) ==(
  for all u in set utilizadores do
   if (u.getEmail() = user.getEmail() or 
     u.getTelefone() = user.getTelefone()) then
    return false;
  return true;
 );
 
 -- Verifica se o trabalho existe por email ou nome ou entidade
 public pure sameTrabalho: Trabalho ==> bool
 sameTrabalho(trab) ==(
  for all u in set trabalhos do
   if (u.getEmail() = trab.getEmail() or 
     u.getNome() = trab.getNome() or 
     u.getEntidade() = trab.getEntidade()) then
    return false;
  return true;
 );
 
functions
-- TODO Define functiones here
traces
-- TODO Define Combinatorial Test Traces here
end Hokify
\end{vdmpp}
\bigskip
\begin{longtable}{|l|r|r|r|}
\hline
Function or operation & Line & Coverage & Calls \\
\hline
\hline
\hyperref[Hokify:18]{Hokify} & 18&100.0\% & 37 \\
\hline
\hyperref[addSkills:22]{addSkills} & 22&100.0\% & 81 \\
\hline
\hyperref[addTrabalhos:28]{addTrabalhos} & 28&100.0\% & 243 \\
\hline
\hyperref[addUtilizadores:22]{addUtilizadores} & 22&100.0\% & 81 \\
\hline
\hyperref[getSkills:34]{getSkills} & 34&100.0\% & 74 \\
\hline
\hyperref[getTrabalhos:41]{getTrabalhos} & 41&100.0\% & 74 \\
\hline
\hyperref[getTrabalhosPorEscolaridade:76]{getTrabalhosPorEscolaridade} & 76&100.0\% & 241 \\
\hline
\hyperref[getTrabalhosPorInteresses:54]{getTrabalhosPorInteresses} & 54&100.0\% & 213 \\
\hline
\hyperref[getTrabalhosPorLocalidade:74]{getTrabalhosPorLocalidade} & 74&100.0\% & 24 \\
\hline
\hyperref[getTrabalhosPorNome:59]{getTrabalhosPorNome} & 59&100.0\% & 212 \\
\hline
\hyperref[getTrabalhosPorPais:74]{getTrabalhosPorPais} & 74&100.0\% & 29 \\
\hline
\hyperref[getTrabalhosPorSkills:64]{getTrabalhosPorSkills} & 64&100.0\% & 213 \\
\hline
\hyperref[getTrabalhosPorUtilizador:86]{getTrabalhosPorUtilizador} & 86&100.0\% & 56 \\
\hline
\hyperref[getUtilizadores:34]{getUtilizadores} & 34&100.0\% & 74 \\
\hline
\hyperref[getUtilizadoresPorEscolaridade:147]{getUtilizadoresPorEscolaridade} & 147&100.0\% & 43 \\
\hline
\hyperref[getUtilizadoresPorInteresses:127]{getUtilizadoresPorInteresses} & 127&100.0\% & 81 \\
\hline
\hyperref[getUtilizadoresPorSkills:137]{getUtilizadoresPorSkills} & 137&100.0\% & 52 \\
\hline
\hyperref[mapEscolaridades:26]{mapEscolaridades} & 26&100.0\% & 243 \\
\hline
\hyperref[sameTrabalho:41]{sameTrabalho} & 41&100.0\% & 28 \\
\hline
\hyperref[sameUser:33]{sameUser} & 33&100.0\% & 41 \\
\hline
\hyperref[searchByName:59]{searchByName} & 59&100.0\% & 1347 \\
\hline
\hline
Hokify.vdmpp & & 100.0\% & 3487 \\
\hline
\end{longtable}

