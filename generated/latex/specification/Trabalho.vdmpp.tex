\begin{vdmpp}[breaklines=true]
class Trabalho
types
-- TODO Define types here
 public String = seq of char;
 public Escolaridade = <Secundario> | <Licenciatura> | <Mestrado> | <Doutoramento>;
 public ListaEscolaridade = set of Escolaridade;
 public Interesses = set of String;
 public Skills = set of String;
 
values
-- TODO Define values here
instance variables
-- TODO Define instance variables here
 private nome: seq of char;
 private entidade: seq of char;
 private email: seq of char;
 private escolaridade: Escolaridade;
 private telefone: nat1;
 private pais: seq of char;
 private localidade: seq of char;
 private interesses: set of String := {};
 private skills: set of String := {};
 private listaEscolaridade: ListaEscolaridade := {};
 
operations
-- TODO Define operations here

 --Construtor
(*@
\label{Trabalho:29}
@*)
 public Trabalho: seq of char * seq of char * seq of char * Escolaridade * nat1 * seq of char * seq of char ==> Trabalho
 Trabalho(nomeC,entidadeC,emailC,escolaridadeC,telefoneC,paisC,localidadeC) == (
  nome := nomeC;
  entidade := entidadeC;
  email := emailC;
  escolaridade := escolaridadeC;
  listaEscolaridades(escolaridade);
  telefone := telefoneC;
  pais := paisC;
  localidade := localidadeC;
  return self;
 )
 post interesses = {} and
   skills = {} and
   nome = nomeC and
   entidade = entidadeC and
   email = emailC and
   escolaridade = escolaridadeC and
   telefone = telefoneC and
   pais = paisC and
   localidade = localidadeC;
 
 -- Adiciona uma lista com as escolaridades no qual possui
(*@
\label{listaEscolaridades:52}
@*)
 public listaEscolaridades: Escolaridade ==> ()
 listaEscolaridades(tipo)==(
 if tipo = <Secundario> then (
  listaEscolaridade := listaEscolaridade union {<Secundario>};
  listaEscolaridade := listaEscolaridade union {<Licenciatura>};
  listaEscolaridade := listaEscolaridade union {<Mestrado>};
  listaEscolaridade := listaEscolaridade union {<Doutoramento>};
 )elseif tipo = <Licenciatura> then(
  listaEscolaridade := listaEscolaridade union {<Licenciatura>};
  listaEscolaridade := listaEscolaridade union {<Mestrado>};
  listaEscolaridade := listaEscolaridade union {<Doutoramento>};
 )elseif tipo = <Mestrado> then(
  listaEscolaridade := listaEscolaridade union {<Mestrado>};
  listaEscolaridade := listaEscolaridade union {<Doutoramento>};
 )else(
  listaEscolaridade := listaEscolaridade union {<Doutoramento>};
 ););

 -- Adicionar interesses
(*@
\label{addInteresse:71}
@*)
 public addInteresse: seq of char ==> ()
 addInteresse(String) == interesses := interesses union {String}
 pre String not in set interesses
 post interesses = interesses~ union {String};
 
 -- Remover interesses
(*@
\label{removeInteresse:77}
@*)
 public removeInteresse: seq of char ==> ()
 removeInteresse(String) == interesses := interesses \ {String}
 pre String in set interesses
 post interesses = interesses~ \ {String};
 
 -- Retorna os interesses
(*@
\label{getInteresses:83}
@*)
 public pure getInteresses : () ==> set of String
 getInteresses() ==
 (
  return interesses;
 );
 
 -- Adicionar skills
(*@
\label{addSkills:90}
@*)
 public addSkills: seq of char ==> ()
 addSkills(String) == skills := skills union {String}
 pre String not in set skills
 post skills = skills~ union {String};
 
 -- Remover skills
(*@
\label{removeSkills:96}
@*)
 public removeSkills: seq of char ==> ()
 removeSkills(String) == skills := skills \ {String}
 pre String in set skills
 post skills = skills~ \ {String};
 
 -- Retorna as skills
(*@
\label{getSkills:102}
@*)
 public pure getSkills : () ==> set of String
 getSkills() ==
 (
  return skills;
 );
 
 -- Retorna o nome
(*@
\label{getNome:109}
@*)
 public pure getNome : () ==> seq of char
 getNome() ==
 (
  return nome;
 );
 -- Retorna o entidade
(*@
\label{getEntidade:115}
@*)
 public pure getEntidade : () ==> seq of char
 getEntidade() ==
 (
  return entidade;
 );
 -- Retorna o email
(*@
\label{getEmail:121}
@*)
 public pure getEmail : () ==> seq of char
 getEmail() ==
 (
  return email;
 );
 -- Retorna o escolaridade
(*@
\label{getEscolaridade:127}
@*)
 public pure getEscolaridade : () ==> Escolaridade
 getEscolaridade() ==
 (
  return escolaridade;
 );
 -- Retorna o telefone
(*@
\label{getTelefone:133}
@*)
 public pure getTelefone : () ==> nat1
 getTelefone() ==
 (
  return telefone;
 );
 -- Retorna o pais
(*@
\label{getPais:139}
@*)
 public pure getPais : () ==> seq of char
 getPais() ==
 (
  return pais;
 );
 -- Retorna o localidade
(*@
\label{getLocalidade:145}
@*)
 public pure getLocalidade : () ==> seq of char
 getLocalidade() ==
 (
  return localidade;
 );
 -- Retorna o listaEscolaridades
(*@
\label{getlistaEscolaridades:151}
@*)
 public pure getlistaEscolaridades : () ==> ListaEscolaridade
 getlistaEscolaridades() ==
 (
  return listaEscolaridade;
 );
 
 -- Editar Nome
(*@
\label{setNome:158}
@*)
 public setNome: seq of char ==> ()
 setNome(newName) == nome := newName
 pre newName <> undefined
 post nome = newName;
 
 -- Editar Entidade
(*@
\label{setEntidade:164}
@*)
 public setEntidade: seq of char ==> ()
 setEntidade(newEntidade) == entidade := newEntidade
 pre newEntidade <> undefined
 post entidade = newEntidade;
 
 -- Editar Email
(*@
\label{setEmail:170}
@*)
 public setEmail: seq of char ==> ()
 setEmail(newEmail) == email := newEmail
 pre newEmail <> undefined
 post email = newEmail;
 
 -- Editar Telefone
(*@
\label{setTelefone:176}
@*)
 public setTelefone: nat1 ==> ()
 setTelefone(newTelefone) == telefone := newTelefone
 pre newTelefone <> undefined
 post telefone = newTelefone;

 -- Editar Pais
(*@
\label{setPais:182}
@*)
 public setPais: seq of char ==> ()
 setPais(newPais) == pais := newPais
 pre newPais <> undefined
 post pais = newPais;
 
 -- Editar Localidade
(*@
\label{setLocalidade:188}
@*)
 public setLocalidade: seq of char ==> ()
 setLocalidade(newLocalidade) == localidade := newLocalidade
 pre newLocalidade <> undefined
 post localidade = newLocalidade;
 
 -- Editar Escolaridade
(*@
\label{setEscolaridade:194}
@*)
 public setEscolaridade: Escolaridade ==> ()
 setEscolaridade(newEscolaridade) == escolaridade := newEscolaridade
 pre newEscolaridade <> undefined
 post escolaridade = newEscolaridade;
 
(*@
\label{nomeSemelhante:199}
@*)
 public pure nomeSemelhante: seq of char ==> bool
 nomeSemelhante(n) == (
  dcl nameS: seq of char := nome;
  dcl found: bool := false;
  
  while len nameS >= len n and not found do (
   found := true;
 
   for index = 1 to len n do
    if found and n(index) <> nameS(index) then (
     found := false;
    );
   
   if found then
    return true
   else (
    nameS := tl nameS;
    found := false;
   );
  );
   
  return false;
 )
 pre len n > 0;
  
functions
-- TODO Define functiones here
traces
-- TODO Define Combinatorial Test Traces here
end Trabalho
\end{vdmpp}
\bigskip
\begin{longtable}{|l|r|r|r|}
\hline
Function or operation & Line & Coverage & Calls \\
\hline
\hline
\hyperref[Trabalho:29]{Trabalho} & 29&100.0\% & 5 \\
\hline
\hyperref[addInteresse:71]{addInteresse} & 71&100.0\% & 4 \\
\hline
\hyperref[addSkills:90]{addSkills} & 90&100.0\% & 4 \\
\hline
\hyperref[getEmail:121]{getEmail} & 121&100.0\% & 14 \\
\hline
\hyperref[getEntidade:115]{getEntidade} & 115&100.0\% & 9 \\
\hline
\hyperref[getEscolaridade:127]{getEscolaridade} & 127&100.0\% & 4 \\
\hline
\hyperref[getInteresses:83]{getInteresses} & 83&100.0\% & 25 \\
\hline
\hyperref[getLocalidade:145]{getLocalidade} & 145&100.0\% & 8 \\
\hline
\hyperref[getNome:109]{getNome} & 109&100.0\% & 10 \\
\hline
\hyperref[getPais:139]{getPais} & 139&100.0\% & 5 \\
\hline
\hyperref[getSkills:102]{getSkills} & 102&100.0\% & 25 \\
\hline
\hyperref[getTelefone:133]{getTelefone} & 133&100.0\% & 2 \\
\hline
\hyperref[getlistaEscolaridades:151]{getlistaEscolaridades} & 151&100.0\% & 18 \\
\hline
\hyperref[listaEscolaridades:52]{listaEscolaridades} & 52&100.0\% & 1 \\
\hline
\hyperref[nomeSemelhante:199]{nomeSemelhante} & 199&100.0\% & 4 \\
\hline
\hyperref[removeInteresse:77]{removeInteresse} & 77&100.0\% & 1 \\
\hline
\hyperref[removeSkills:96]{removeSkills} & 96&100.0\% & 2 \\
\hline
\hyperref[setEmail:170]{setEmail} & 170&100.0\% & 1 \\
\hline
\hyperref[setEntidade:164]{setEntidade} & 164&100.0\% & 2 \\
\hline
\hyperref[setEscolaridade:194]{setEscolaridade} & 194&100.0\% & 1 \\
\hline
\hyperref[setLocalidade:188]{setLocalidade} & 188&100.0\% & 1 \\
\hline
\hyperref[setNome:158]{setNome} & 158&100.0\% & 1 \\
\hline
\hyperref[setPais:182]{setPais} & 182&100.0\% & 1 \\
\hline
\hyperref[setTelefone:176]{setTelefone} & 176&100.0\% & 1 \\
\hline
\hline
Trabalho.vdmpp & & 100.0\% & 149 \\
\hline
\end{longtable}

