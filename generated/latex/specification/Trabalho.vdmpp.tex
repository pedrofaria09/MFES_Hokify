\begin{vdmpp}[breaklines=true]
class Trabalho
types
-- TODO Define types here
 public String = seq of char;
 public Escolaridade = <Secundario> | <Licenciatura> | <Mestrado> | <Doutoramento>;
 public ListaEscolaridade = set of Escolaridade;
 public Interesses = set of String;
 public Skills = set of String;
 
values
-- TODO Define values here
instance variables
-- TODO Define instance variables here
 private nome: seq of char;
 private entidade: seq of char;
 private email: seq of char;
 private escolaridade: Escolaridade;
 private telefone: nat1;
 private pais: seq of char;
 private localidade: seq of char;
 private interesses: set of String := {};
 private skills: set of String := {};
 private listaEscolaridade: ListaEscolaridade := {};
 
operations
-- TODO Define operations here

 --Construtor
(*@
\label{Trabalho:29}
@*)
 public Trabalho: seq of char * seq of char * seq of char * Escolaridade * nat1 * seq of char * seq of char ==> Trabalho
 Trabalho(nomeC,entidadeC,emailC,escolaridadeC,telefoneC,paisC,localidadeC) == (
  nome := nomeC;
  entidade := entidadeC;
  email := emailC;
  escolaridade := escolaridadeC;
  listaEscolaridades(escolaridade);
  telefone := telefoneC;
  pais := paisC;
  localidade := localidadeC;
  return self;
 )
 post interesses = {} and
   skills = {} and
   nome = nomeC and
   entidade = entidadeC and
   email = emailC and
   escolaridade = escolaridadeC and
   telefone = telefoneC and
   pais = paisC and
   localidade = localidadeC;
  
(*@
\label{listaEscolaridades:51}
@*)
 public listaEscolaridades: Escolaridade ==> ()
 listaEscolaridades(tipo)==(
 if tipo = <Secundario> then (
  listaEscolaridade := listaEscolaridade union {<Secundario>};
  listaEscolaridade := listaEscolaridade union {<Licenciatura>};
  listaEscolaridade := listaEscolaridade union {<Mestrado>};
  listaEscolaridade := listaEscolaridade union {<Doutoramento>};
 )elseif tipo = <Licenciatura> then(
  listaEscolaridade := listaEscolaridade union {<Licenciatura>};
  listaEscolaridade := listaEscolaridade union {<Mestrado>};
  listaEscolaridade := listaEscolaridade union {<Doutoramento>};
 )elseif tipo = <Mestrado> then(
  listaEscolaridade := listaEscolaridade union {<Mestrado>};
  listaEscolaridade := listaEscolaridade union {<Doutoramento>};
 )else(
  listaEscolaridade := listaEscolaridade union {<Doutoramento>};
 ););

 -- Adicionar interesses
(*@
\label{addInteresse:70}
@*)
 public addInteresse: seq of char ==> ()
 addInteresse(String) == interesses := interesses union {String}
 pre String not in set interesses
 post interesses = interesses~ union {String};
 
 -- Remover interesses
(*@
\label{removeInteresse:76}
@*)
 public removeInteresse: seq of char ==> ()
 removeInteresse(String) == interesses := interesses \ {String}
 pre String in set interesses
 post interesses = interesses~ \ {String};
 
 -- Retorna os interesses
(*@
\label{getInteresses:82}
@*)
 public pure getInteresses : () ==> set of String
 getInteresses() ==
 (
  return interesses;
 );
 
 -- Adicionar skills
(*@
\label{addSkills:89}
@*)
 public addSkills: seq of char ==> ()
 addSkills(String) == skills := skills union {String}
 pre String not in set skills
 post skills = skills~ union {String};
 
 -- Remover skills
(*@
\label{removeSkills:95}
@*)
 public removeSkills: seq of char ==> ()
 removeSkills(String) == skills := skills \ {String}
 pre String in set skills
 post skills = skills~ \ {String};
 
 -- Retorna as skills
(*@
\label{getSkills:101}
@*)
 public pure getSkills : () ==> set of String
 getSkills() ==
 (
  return skills;
 );
 
 -- Retorna o nome
(*@
\label{getNome:108}
@*)
 public pure getNome : () ==> seq of char
 getNome() ==
 (
  return nome;
 );
 -- Retorna o entidade
(*@
\label{getEntidade:114}
@*)
 public pure getEntidade : () ==> seq of char
 getEntidade() ==
 (
  return entidade;
 );
 -- Retorna o email
(*@
\label{getEmail:120}
@*)
 public pure getEmail : () ==> seq of char
 getEmail() ==
 (
  return email;
 );
 -- Retorna o escolaridade
(*@
\label{getEscolaridade:126}
@*)
 public pure getEscolaridade : () ==> Escolaridade
 getEscolaridade() ==
 (
  return escolaridade;
 );
 -- Retorna o telefone
(*@
\label{getTelefone:132}
@*)
 public pure getTelefone : () ==> nat1
 getTelefone() ==
 (
  return telefone;
 );
 -- Retorna o pais
(*@
\label{getPais:138}
@*)
 public pure getPais : () ==> seq of char
 getPais() ==
 (
  return pais;
 );
 -- Retorna o localidade
(*@
\label{getLocalidade:144}
@*)
 public pure getLocalidade : () ==> seq of char
 getLocalidade() ==
 (
  return localidade;
 );
 -- Retorna o listaEscolaridades
(*@
\label{getlistaEscolaridades:150}
@*)
 public pure getlistaEscolaridades : () ==> ListaEscolaridade
 getlistaEscolaridades() ==
 (
  return listaEscolaridade;
 );
 
 -- Editar Nome
(*@
\label{setNome:157}
@*)
 public setNome: seq of char ==> ()
 setNome(newName) == nome := newName
 pre newName <> undefined
 post nome = newName;
 
 -- Editar Entidade
(*@
\label{setEntidade:163}
@*)
 public setEntidade: seq of char ==> ()
 setEntidade(newEntidade) == entidade := newEntidade
 pre newEntidade <> undefined
 post entidade = newEntidade;
 
 -- Editar Email
(*@
\label{setEmail:169}
@*)
 public setEmail: seq of char ==> ()
 setEmail(newEmail) == email := newEmail
 pre newEmail <> undefined
 post email = newEmail;
 
 -- Editar Telefone
(*@
\label{setTelefone:175}
@*)
 public setTelefone: nat1 ==> ()
 setTelefone(newTelefone) == telefone := newTelefone
 pre newTelefone <> undefined
 post telefone = newTelefone;

 -- Editar Pais
(*@
\label{setPais:181}
@*)
 public setPais: seq of char ==> ()
 setPais(newPais) == pais := newPais
 pre newPais <> undefined
 post pais = newPais;
 
 -- Editar Localidade
(*@
\label{setLocalidade:187}
@*)
 public setLocalidade: seq of char ==> ()
 setLocalidade(newLocalidade) == localidade := newLocalidade
 pre newLocalidade <> undefined
 post localidade = newLocalidade;
 
 -- Editar Escolaridade
(*@
\label{setEscolaridade:193}
@*)
 public setEscolaridade: Escolaridade ==> ()
 setEscolaridade(newEscolaridade) == escolaridade := newEscolaridade
 pre newEscolaridade <> undefined
 post escolaridade = newEscolaridade;
 
(*@
\label{nomeSemelhante:198}
@*)
 public pure nomeSemelhante: seq of char ==> bool
 nomeSemelhante(n) == (
  dcl nameS: seq of char := nome;
  dcl found: bool := false;
  
  while len nameS >= len n and not found do (
   found := true;
 
   for index = 1 to len n do
    if found and n(index) <> nameS(index) then (
     found := false;
    );
   
   if found then
    return true
   else (
    nameS := tl nameS;
    found := false;
   );
  );
   
  return false;
 )
 pre len n > 0;
  
functions
-- TODO Define functiones here
traces
-- TODO Define Combinatorial Test Traces here
end Trabalho
\end{vdmpp}
\bigskip
\begin{longtable}{|l|r|r|r|}
\hline
Function or operation & Line & Coverage & Calls \\
\hline
\hline
\hyperref[Trabalho:29]{Trabalho} & 29&100.0\% & 20 \\
\hline
\hyperref[addInteresse:70]{addInteresse} & 70&100.0\% & 16 \\
\hline
\hyperref[addSkills:89]{addSkills} & 89&100.0\% & 16 \\
\hline
\hyperref[getEmail:120]{getEmail} & 120&100.0\% & 44 \\
\hline
\hyperref[getEntidade:114]{getEntidade} & 114&100.0\% & 30 \\
\hline
\hyperref[getEscolaridade:126]{getEscolaridade} & 126&100.0\% & 16 \\
\hline
\hyperref[getInteresses:82]{getInteresses} & 82&100.0\% & 100 \\
\hline
\hyperref[getLocalidade:144]{getLocalidade} & 144&100.0\% & 32 \\
\hline
\hyperref[getNome:108]{getNome} & 108&100.0\% & 34 \\
\hline
\hyperref[getPais:138]{getPais} & 138&100.0\% & 20 \\
\hline
\hyperref[getSkills:101]{getSkills} & 101&100.0\% & 100 \\
\hline
\hyperref[getTelefone:132]{getTelefone} & 132&100.0\% & 8 \\
\hline
\hyperref[getlistaEscolaridades:150]{getlistaEscolaridades} & 150&100.0\% & 72 \\
\hline
\hyperref[listaEscolaridades:51]{listaEscolaridades} & 51&100.0\% & 9 \\
\hline
\hyperref[nomeSemelhante:198]{nomeSemelhante} & 198&100.0\% & 36 \\
\hline
\hyperref[removeInteresse:76]{removeInteresse} & 76&100.0\% & 4 \\
\hline
\hyperref[removeSkills:95]{removeSkills} & 95&100.0\% & 4 \\
\hline
\hyperref[setEmail:169]{setEmail} & 169&100.0\% & 4 \\
\hline
\hyperref[setEntidade:163]{setEntidade} & 163&100.0\% & 4 \\
\hline
\hyperref[setEscolaridade:193]{setEscolaridade} & 193&100.0\% & 4 \\
\hline
\hyperref[setLocalidade:187]{setLocalidade} & 187&100.0\% & 4 \\
\hline
\hyperref[setNome:157]{setNome} & 157&100.0\% & 4 \\
\hline
\hyperref[setPais:181]{setPais} & 181&100.0\% & 4 \\
\hline
\hyperref[setTelefone:175]{setTelefone} & 175&100.0\% & 4 \\
\hline
\hline
Trabalho.vdmpp & & 100.0\% & 589 \\
\hline
\end{longtable}

